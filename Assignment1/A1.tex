\documentclass[12pt]{article}
\usepackage{amsmath,amsthm,amssymb,amsfonts}
\usepackage{enumerate}
\newenvironment{solution}
{\begin{center}
        \begin{tabular}
		{|p{0.9\textwidth}|}
        \hline\\
    \textbf{Solution: }\\
	}
    { 
        \\\\\hline
    \end{tabular} 
    \end{center}
    }
\newcounter{problem}[section]
\newenvironment{problem}[1][]{\refstepcounter{problem}
\par\medskip
   \noindent \textbf{Problem~\theproblem.\\ #1} \rmfamily}{\medskip}

\begin{document}

\title{\textbf{Assignment 1}}
\author{CMPT 215\\
\textbf{Due}: July $13^{th}$, $2017$} 
\date{}

\maketitle 
\textbf{Total: 50 marks}
\begin{problem}
    (8 marks) Define and write the equations for the following metrics in terms of instruction count ($I$), number of clock cycles ($C$) and clock cycle time ($T$).
    \begin{enumerate}[i]
        \item Frequency
        \item CPI
        \item MIPS
        \item CPU execution time
    \end{enumerate}
    
\end{problem}
\clearpage
\begin{table}[!h]
        \label{table:1}
        \begin{tabular}{|p{3cm}||p{2cm}|p{2cm}|p{2cm}|}
            \hline
            \multicolumn{4}{|c|}{Computer System Metrics}
            \\
            \hline
            Instruction type & CPI & Frequency & Clock cycle($ns$)\\
            \hline
            A & $1$ & $20\%$ & $0.5$ \\
            \hline
            B & $2$ & $40\%$ & $0.5$ \\
            \hline
            C & $3$ & $30\%$ & $0.5$ \\
            \hline
            D & $4$ & $10\%$ & $0.5$ \\  
            \hline
        \end{tabular}
    \end{table}

    \begin{problem}
(6 marks) Calculate the following performance metrics using the given values in the table above and showing all \textbf{steps and units}.
\begin{enumerate}[i]
    \item CPI 
    \item MIPS 
    \item CPU execution time for $500$ instructions
\end{enumerate}
\end{problem}

\begin{problem}
    (4 marks) Suppose it is possible to third ($1/3$) the number of type C instructions in the table above, what will be the new CPI value?
\end{problem}

\begin{problem}
    (4 marks) Describe and write the equation for \textit{Ambdals law}.
\end{problem}
\clearpage
\begin{problem}
    (10 marks) Convert the following integers into binary and hexadecimal.
    \begin{enumerate}[i]
        \item $27$
        \item $400$
        \item $88$
        \item $10$
        \item $99$
    \end{enumerate}
\end{problem}

\begin{problem}
    (5 marks) Convert the following binary to their integer values.
    \begin{enumerate}[i]
        \item 1001
        \item 101110
        \item 111001
        \item 01100011
        \item 01110010
    \end{enumerate}
\end{problem}

\begin{problem}
    (3 marks) What is the difference between 1's compliment and 2's compliment?
    When are they used and what applications are they used in?
\end{problem}

\begin{problem}
    (10 marks) Convert the following to 8-bit 1's compliment and 2'compliment.
    \begin{enumerate}[i]
        \item 20
        \item 0
        \item -100
        \item 127
        \item -127
    \end{enumerate}
\end{problem}


\textbf{Bonus:}
\begin{problem}
    (3 marks) Name the top three fastest machines in the world and mention which benchmark was used to test their performance. 
\end{problem}

\end{document}
